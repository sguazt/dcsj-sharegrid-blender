%%
%% Copyright (C) 2008  Distributed Computing System (DCS) Group, Computer
%% Science Department - University of Piemonte Orientale, Alessandria (Italy).
%%
%% This program is free software: you can redistribute it and/or modify
%% it under the terms of the GNU Lesser General Public License as published
%% by the Free Software Foundation, either version 3 of the License, or
%% (at your option) any later version.
%%
%% This program is distributed in the hope that it will be useful,
%% but WITHOUT ANY WARRANTY; without even the implied warranty of
%% MERCHANTABILITY or FITNESS FOR A PARTICULAR PURPOSE.  See the
%% GNU Lesser General Public License for more details.
%%
%% You should have received a copy of the GNU Lesser General Public License
%% along with this program.  If not, see <http://www.gnu.org/licenses/>.
%%

\section{Installazione} \label{sec:inst}

Questa sezione descrive i passi necessari per installare l'applicazione \mgTheApp{}.

\subsection{Requisiti di Sistema} \label{ssec:inst-sysreq}

Per poter eseguire l'applicazione \mgTheApp{} \`e necessario che il proprio sistema disponga dei seguenti requisiti software:
\begin{itemize}
\item \emph{Java Runtime Environment} (\emph{JRE}) versione $1.6$, scaricabile, ad esempio, dal sito della \emph{Sun} (\href{http://java.sun.com}{http://java.sun.com}).
\item Componente \emph{mygrid} del middleware \emph{OurGrid}, versione $3.3.1$, scaricabile dal sito di \emph{OurGrid} (\href{http://www.ourgrid.org}{http://www.ourgrid.org}) e disponibile anche sul sito di \emph{ShareGrid} (\href{http://dcs.di.unipmn.it}{http://dcs.di.unipmn.it}).
\item Sistema operativo Unix-like, come, ad esempio, Linux (per soddisfare i requisiti del componente \emph{mygrid} di \emph{OurGrid}).
\item Programma in grado di scompattare file in formato \emph{ZIP}.
\end{itemize}

Per verificare che la versione del JRE installato nel proprio sistema sia quella corretta, eseguire, da un prompt dei comandi, il seguente comando:
\begin{mgCodeBox}
java -version
\end{mgCodeBox}
Dovrebbe apparire un messaggio mostrante la versione del JRE.
Per esempio, utilizzando il \emph{Sun JRE $1.6$ Update $2$}, il messaggio visualizzato, dopo l'esecuzione del comando precedentemente mostrato, dovrebbe essere simile al seguente:
\begin{mgCodeBox}
java version ``1.6.0\_02''\newline
Java(TM) SE Runtime Environment (build 1.6.0\_02-b05)\newline
Java HotSpot(TM) Client VM (build 1.6.0\_02-b05, mixed mode, sharing)
\end{mgCodeBox}

Per controllare che la versione del componente \emph{mygrid} del middleware \emph{OurGrid} sia quella richiesta, eseguire, da un prompt dei comandi, le seguenti istruzioni:
\begin{mgCodeBox}
cd /path/to/mygrid/installation\newline
./bin/mygrid version
\end{mgCodeBox}
Per esempio, nel caso in cui \emph{mygrid} sia stato scaricato dal sito di \emph{Sharegrid}, i comandi da eseguire sarebbero:
\begin{mgCodeBox}
cd shareGrid/clientSG\newline
./bin/mygrid version
\end{mgCodeBox}
L'output del comando dovrebbe essere simile al seguente:
\begin{mgCodeBox}
OurGrid 3.3.1 - MyGrid\newline
\newline
For MyGrid updates and additional information, see the\newline
OurGrid Project home page at http://www.ourgrid.org/ 
\end{mgCodeBox}

\`E inoltre indispensabile avere una connessione Internet per poter accedere all'infrastruttura Grid di \emph{ShareGrid}.

\subsection{Installazione} \label{sec:inst-inst}

Se i requisiti di sistema (\S \ref{ssec:inst-sysreq}) sono soddisfatti \`e possibile procedere all'installazione di \mgTheApp{}.
Al fine di semplificare la descrizione della procedura di installazione, nel seguito si effettueranno le seguenti assunzioni:
\begin{itemize}
\item L'utente ha installato il componente \emph{mygrid} attraverso il file e le istruzioni pubblicate sul sito di \emph{ShareGrid}; ne segue, che sul disco dell'utente \`e presente la cartella \emph{shareGrid}, creata durante il processo di installazione di \emph{mygrid}.
\item Il programma utilizzato dall'utente per scompattare i file in formato ZIP \`e \emph{unzip}.
\end{itemize}

Di seguito viene presentata la procedura di installazione di \mgTheApp{}.
\begin{enumerate}
\item Scaricare dal sito di \emph{ShareGrid} (\href{http://dcs.di.unipmn.it}{http://dcs.di.unipmn.it}) il file \mgTheAppFile{.zip}.
\item Scompattare il file all'interno della cartella \emph{shareGrid} creata durante l'installazione del componente \emph{mygrid}:
\begin{mgCodeBox}
cd shareGrid \newline
unzip \mgTheAppFile{.zip}
\end{mgCodeBox}
Se non si sono verificati errori, dovrebbe essere stata creata la cartella \mgTheApp{}.
\item Spostarsi nella directory \mgTheApp{}
\begin{mgCodeBox}
cd \mgTheApp{}
\end{mgCodeBox}
\item Eseguire lo script di installazione \emph{install.sh}
\begin{mgCodeBox}
./install.sh
\end{mgCodeBox}
\begin{enumerate}
\item Se l'esecuzione dello script termina con il messaggio:
\begin{mgCodeBox}
The path '/full/path/to/shareGrid/clientSG' seems to work! \newline
Installation succeded! \newline
Bye!!
\end{mgCodeBox}
allora \`e possibile proseguire con l'installazione.
\item Altrimenti, nel caso in cui venga mostrato il messaggio:
\begin{mgCodeBox}
!! WARNING !! \newline
MGROOT\_GUESS variable is set to a non existent path (\dots)! \newline
Please, set it to the MyGrid installation path. \newline
Installation failed. Sorry! \newline
Bye!!
\end{mgCodeBox}
editare il file \emph{install.sh} e impostare la variabile di ambiente \texttt{MGROOT\_GUESS} in modo che contenga il percorso completo della cartella in cui \`e stato installato il componente \emph{mygrid}. Per le assunzioni fatte in precedenza, tale percorso dovrebbe essere dato da \emph{/full/path/to/shareGrid/clientSG}: 
\begin{mgCodeBox}
MGROOT\_GUESS=/full/path/to/shareGrid/clientSG
\end{mgCodeBox}
Ripetere quindi il comando mostrato all'inizio di questo passo.
\end{enumerate}
\item Se l'esecuzione di \emph{install.sh} termina con successo, dovrebbe essere presente nella stessa cartella il nuovo file \mgTheAppFile{.sh}.
\item A questo punto, l'installazione di \mgTheApp{} \`e completata.
\end{enumerate}
